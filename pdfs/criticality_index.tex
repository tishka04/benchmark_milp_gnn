% ============================================================
% Preamble for Scenario Criticality Indicator Section
% ============================================================

\documentclass[11pt,a4paper]{article}

% ------------------------------------------------------------
% Encoding and language
% ------------------------------------------------------------
\usepackage[T1]{fontenc}
\usepackage[utf8]{inputenc}   % remove if using lualatex/xelatex
\usepackage[english]{babel}

% ------------------------------------------------------------
% Geometry and layout
% ------------------------------------------------------------
\usepackage{geometry}
\geometry{margin=2.2cm}

% ------------------------------------------------------------
% Mathematics and symbols
% ------------------------------------------------------------
\usepackage{amsmath,amssymb,amsfonts}
\usepackage{bm}               % bold math symbols

% ------------------------------------------------------------
% Tables and lists
% ------------------------------------------------------------
\usepackage{booktabs}         % professional tables
\usepackage{enumitem}         % custom enumerate/itemize

% ------------------------------------------------------------
% Colors and boxes
% ------------------------------------------------------------
\usepackage{xcolor}
\definecolor{lightgray}{gray}{0.96}

\usepackage{tcolorbox}
\tcbset{
  colback=lightgray,
  colframe=black!30,
  boxrule=0.6pt,
  arc=2pt,
  left=6pt,
  right=6pt,
  top=4pt,
  bottom=4pt
}

% ------------------------------------------------------------
% Hyperlinks (recommended)
% ------------------------------------------------------------
\usepackage{hyperref}
\hypersetup{
  colorlinks=true,
  linkcolor=blue!50!black,
  citecolor=blue!50!black,
  urlcolor=blue!50!black
}

% ------------------------------------------------------------
% Figures (if needed later)
% ------------------------------------------------------------
\usepackage{graphicx}

% ------------------------------------------------------------
% Floating control (optional but useful)
% ------------------------------------------------------------
\usepackage{float}

% ------------------------------------------------------------
% Micro-typography (recommended for final submission)
% ------------------------------------------------------------
\usepackage{microtype}

% ------------------------------------------------------------
% Title info (example)
% ------------------------------------------------------------
\title{Scenario Criticality Metrics for Large-Scale Power System Flexibility Analysis}
\author{Th\'eotime Coudray}
\date{\today}

% ============================================================
% Document start
% ============================================================
\begin{document}
\maketitle


% ============================================================
% Criticality indicator: Physical Stress + Combinatorial Hardness
% (22 components, grouped by category)
% ============================================================

\subsection{Scenario Criticality Indicator}

We define an \emph{ex ante} scenario criticality indicator by decomposing difficulty into:
(i) a \textbf{physical stress} component capturing techno-operational tension, and
(ii) a \textbf{combinatorial hardness} component capturing structural MILP difficulty.
All metrics are computable from scenario inputs prior to solving.

\begin{tcolorbox}[colback=gray!4,colframe=black!25,title=Final Form]
\begin{equation}
\label{eq:criticality_final}
\boxed{
\mathrm{Crit}(s)
=
\alpha \,\mathrm{Stress}(s)
+
(1-\alpha)\,\mathrm{Hard}(s)
}
\qquad \text{with } \alpha \in [0,1].
\end{equation}
\end{tcolorbox}

\paragraph{Notation.}
Let $t\in\mathcal{T}$ denote time steps, $z\in\mathcal{Z}$ zones, $\ell\in\mathcal{L}$ transmission lines.
Demand is $D(t)$ and residual demand is $D_{\mathrm{res}}(t)=D(t)-P^{\mathrm{VRE}}_{\mathrm{avail}}(t)$.
Peak zonal demand is $D^{\mathrm{peak}}_z=\max_{t} D_z(t)$.
Installed/available capacities are denoted $P^{(\cdot)}$ (power) and $E^{(\cdot)}$ (energy).

\subsubsection{A. Physical Stress Index $\mathrm{Stress}(s)$}

\paragraph{A.1 Variability and intermittence (exogenous drivers).}
\begin{enumerate}[label=\textbf{A\arabic*.}, leftmargin=*, itemsep=4pt]
  \item \textbf{VRE penetration ratio}
  \begin{equation}
  \label{eq:stress_vre_pen}
  \mathrm{VREPen}(s)
  =
  \frac{\sum_{t\in\mathcal{T}} P^{\mathrm{VRE}}_{\mathrm{avail}}(t)}
       {\sum_{t\in\mathcal{T}} D(t)}.
  \end{equation}

  \item \textbf{Net (residual) demand volatility}
  \begin{equation}
  \label{eq:stress_net_vol}
  \mathrm{Vol}_{\mathrm{res}}(s)
  =
  \mathrm{std}_{t\in\mathcal{T}}\!\left(D_{\mathrm{res}}(t)\right).
  \end{equation}

  \item \textbf{Peak-to-valley ratio of residual load}
  \begin{equation}
  \label{eq:stress_ptv}
  \mathrm{PTV}_{\mathrm{res}}(s)
  =
  \frac{\max_{t\in\mathcal{T}} D_{\mathrm{res}}(t)}
       {\min_{t\in\mathcal{T}} D_{\mathrm{res}}(t)}.
  \end{equation}

  \item \textbf{Short-term variability index (first differences)}
  \begin{equation}
  \label{eq:stress_short_var}
  \mathrm{STVar}(s)
  =
  \mathrm{std}_{t\in\mathcal{T}\setminus\{t_0\}}
  \!\left(D_{\mathrm{res}}(t)-D_{\mathrm{res}}(t-1)\right).
  \end{equation}
\end{enumerate}

\paragraph{A.2 Load level and adequacy margins.}
\begin{enumerate}[label=\textbf{A\arabic*.}, leftmargin=*, resume, itemsep=4pt]
  \item \textbf{Demand scale factor}
  \begin{equation}
  \label{eq:stress_demand_scale}
  \mathrm{DScale}(s)
  =
  \frac{\sum_{t\in\mathcal{T}} D(t)}
       {\sum_{t\in\mathcal{T}} D_{\mathrm{baseline}}(t)}.
  \end{equation}

  \item \textbf{Peak load to firm capacity ratio}
  \begin{equation}
  \label{eq:stress_peak_firm}
  \mathrm{PeakFirm}(s)
  =
  \frac{\max_{t\in\mathcal{T}} D(t)}
       {P^{\mathrm{firm}}_{\mathrm{avail}}}.
  \end{equation}
\end{enumerate}

\paragraph{A.3 Flexibility resource adequacy (inverse/headroom).}
\begin{enumerate}[label=\textbf{A\arabic*.}, leftmargin=*, resume, itemsep=4pt]
  \item \textbf{Storage power adequacy (inverse)}
  \begin{equation}
  \label{eq:stress_stor_power_inv}
  \mathrm{InvStorP}(s)
  =
  1 - \frac{P^{\mathrm{stor}}_{\max}}
           {\max_{t\in\mathcal{T}} D_{\mathrm{res}}(t)}.
  \end{equation}

  \item \textbf{Storage energy adequacy (inverse)}
  \begin{equation}
  \label{eq:stress_stor_energy_inv}
  \mathrm{InvStorE}(s)
  =
  1 - \frac{E^{\mathrm{stor}}_{\max}}
           {\sum_{t\in\mathcal{T}} \left|D^{+}_{\mathrm{res}}(t)\right|},
  \qquad
  D^{+}_{\mathrm{res}}(t)=\max\{D_{\mathrm{res}}(t),0\}.
  \end{equation}

  \item \textbf{Thermal flexibility ratio (inverse)}
  \begin{equation}
  \label{eq:stress_thermal_flex_inv}
  \mathrm{InvThFlex}(s)
  =
  1 - \frac{\sum P^{\mathrm{th}}_{\mathrm{flex}}}
           {\max_{t\in\mathcal{T}} D_{\mathrm{res}}(t)}.
  \end{equation}

  \item \textbf{Demand-response headroom (inverse)}
  \begin{equation}
  \label{eq:stress_dr_headroom_inv}
  \mathrm{InvDR}(s)
  =
  1 - \frac{\sum_{t\in\mathcal{T}} DR^{\max}(t)}
           {\sum_{t\in\mathcal{T}} D(t)}.
  \end{equation}
\end{enumerate}

\paragraph{A.4 Spatial tension (physical).}
\begin{enumerate}[label=\textbf{A\arabic*.}, leftmargin=*, resume, itemsep=4pt]
  \item \textbf{Import/export reliance}
  \begin{equation}
  \label{eq:stress_trade_reliance}
  \mathrm{TradeRel}(s)
  =
  \frac{\sum_{t\in\mathcal{T}} | \mathrm{Import}(t) |}
       {\sum_{t\in\mathcal{T}} D(t)}.
  \end{equation}

  \item \textbf{Congestion exposure proxy (inverse)}
  \begin{equation}
  \label{eq:stress_cong_proxy}
  \mathrm{InvCong}(s)
  =
  1 - \frac{\sum_{\ell\in\mathcal{L}} F^{\max}_{\ell}}
           {\sum_{z\in\mathcal{Z}} D^{\mathrm{peak}}_{z}}.
  \end{equation}
\end{enumerate}

\subsubsection{B. Combinatorial Hardness Index $\mathrm{Hard}(s)$}

\paragraph{B.1 Problem size (scaling).}
\begin{enumerate}[label=\textbf{B\arabic*.}, leftmargin=*, itemsep=4pt]
  \item \textbf{Number of zones}
  \begin{equation}
  \label{eq:hard_nzones}
  \mathrm{NZones}(s)=|\mathcal{Z}|.
  \end{equation}

  \item \textbf{Horizon length}
  \begin{equation}
  \label{eq:hard_horizon}
  \mathrm{Horizon}(s)=|\mathcal{T}|.
  \end{equation}

  \item \textbf{Estimated number of binary variables}
  \begin{equation}
  \label{eq:hard_bin_est}
  \mathrm{BinEst}(s)
  \approx
  |\mathcal{T}|\cdot\Big(N^{\mathrm{th}} + N^{\mathrm{nuc}} + N^{\mathrm{DRflags}}\Big).
  \end{equation}
\end{enumerate}

\paragraph{B.2 Structural non-convexity.}
\begin{enumerate}[label=\textbf{B\arabic*.}, leftmargin=*, resume, itemsep=4pt]
  \item \textbf{Thermal unit density}
  \begin{equation}
  \label{eq:hard_th_density}
  \mathrm{ThDens}(s)
  =
  \frac{N^{\mathrm{th}}}{|\mathcal{Z}|}.
  \end{equation}

  \item \textbf{Tight minimum-generation ratio}
  \begin{equation}
  \label{eq:hard_mingen}
  \mathrm{MinGenTight}(s)
  =
  \frac{P^{\mathrm{th}}_{\min}}{P^{\mathrm{th}}_{\max}}.
  \end{equation}

  \item \textbf{Startup cost intensity}
  \begin{equation}
  \label{eq:hard_startup_intensity}
  \mathrm{StartInt}(s)
  =
  \frac{\sum C^{\mathrm{startup}}}{\sum C^{\mathrm{var}}}.
  \end{equation}
\end{enumerate}

\paragraph{B.3 Temporal coupling (intertemporal hardness).}
\begin{enumerate}[label=\textbf{B\arabic*.}, leftmargin=*, resume, itemsep=4pt]
  \item \textbf{Ramping tightness index}
  \begin{equation}
  \label{eq:hard_ramp_tight}
  \mathrm{RampTight}(s)
  =
  \frac{\sum P^{\mathrm{th}}_{\max}}{\sum R^{\mathrm{th}}}.
  \end{equation}

  \item \textbf{Storage SOC tightness proxy}
  \begin{equation}
  \label{eq:hard_soc_tight}
  \mathrm{SOCTight}(s)
  =
  \left(\frac{E^{\mathrm{stor}}_{\max}}{P^{\mathrm{stor}}_{\max}}\right)
  \cdot
  \left( \frac{1}{\tau_{\mathrm{SOC}}} \right),
  \end{equation}
  where $\tau_{\mathrm{SOC}}$ is the final SOC tolerance (smaller $\Rightarrow$ tighter).
\end{enumerate}

\paragraph{B.4 Spatial coupling (network-induced hardness).}
\begin{enumerate}[label=\textbf{B\arabic*.}, leftmargin=*, resume, itemsep=4pt]
  \item \textbf{Interconnection density}
  \begin{equation}
  \label{eq:hard_interconn_density}
  \mathrm{InterDens}(s)
  =
  \frac{|\mathcal{L}|}{|\mathcal{Z}|}.
  \end{equation}

  \item \textbf{Network heterogeneity index}
  \begin{equation}
  \label{eq:hard_net_hetero}
  \mathrm{NetHet}(s)
  =
  \mathrm{Var}\big(\{F^{\max}_{\ell}\}_{\ell\in\mathcal{L}}\big)
  \;\;\text{or}\;\;
  \mathrm{Var}\big(\{\deg(z)\}_{z\in\mathcal{Z}}\big).
  \end{equation}
\end{enumerate}

\subsubsection{Normalization and Aggregation}

Each metric above is normalized to $[0,1]$ using either (i) dataset-based min--max scaling, or
(ii) generator-bound scaling (preferred for stability), then combined as weighted sums:
\begin{equation}
\label{eq:stress_agg}
\mathrm{Stress}(s)=\sum_{i=1}^{12} w^{(S)}_i \,\tilde{x}^{(S)}_i(s),
\qquad
\mathrm{Hard}(s)=\sum_{j=1}^{10} w^{(H)}_j \,\tilde{x}^{(H)}_j(s),
\end{equation}
with $\sum_i w^{(S)}_i = 1$, $\sum_j w^{(H)}_j = 1$.

% ------------------------------------------------------------
% Optional: compact table listing all 22 components (no formulas)
% ------------------------------------------------------------
\begin{table}[h]
\centering
\caption{Components of the scenario criticality indicator (22 metrics).}
\label{tab:criticality_components}
\begin{tabular}{ll}
\toprule
\textbf{Category} & \textbf{Metric} \\
\midrule
A.1 Variability & (A1) VRE penetration; (A2) residual volatility; (A3) residual peak-to-valley; (A4) short-term variability \\
A.2 Load/Margins & (A5) demand scale factor; (A6) peak-to-firm capacity ratio \\
A.3 Flexibility & (A7) inverse storage power adequacy; (A8) inverse storage energy adequacy; (A9) inverse thermal flexibility; (A10) inverse DR headroom \\
A.4 Spatial tension & (A11) trade reliance; (A12) inverse congestion exposure proxy \\
B.1 Size & (B1) number of zones; (B2) horizon length; (B3) estimated binary count \\
B.2 Non-convexity & (B4) thermal density; (B5) min-gen tightness; (B6) startup cost intensity \\
B.3 Temporal coupling & (B7) ramping tightness; (B8) SOC tightness proxy \\
B.4 Spatial coupling & (B9) interconnection density; (B10) network heterogeneity index \\
\bottomrule
\end{tabular}
\end{table}


\end{document}