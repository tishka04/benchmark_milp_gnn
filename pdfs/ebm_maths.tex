\documentclass[11pt]{article}

\usepackage[T1]{fontenc}
\usepackage[utf8]{inputenc}
\usepackage{amsmath, amssymb, amsfonts}
\usepackage{geometry}
\geometry{margin=2.5cm}
\usepackage{hyperref}
\usepackage{bm}

\title{Technical Note:\\
Energy-Based Model for UC/DR/Storage Configurations\\
Conditioned on 128-Dimensional Graph Embeddings}
\author{}
\date{}

\begin{document}
\maketitle

\section*{1. Problem Setting}

For each scenario $t$, we dispose of:
\begin{itemize}
    \item A graph-derived embedding $h_t \in \mathbb{R}^{128}$.
    \item A binary decision vector 
    \[
        u_t \in \{0,1\}^N,
    \]
    representing UC flags, DR activations, and discrete storage choices.
\end{itemize}

We aim to learn an Energy-Based Model (EBM) of the form:
\[
    E_\theta(u \mid h) \in \mathbb{R},
\]
such that MILP-feasible configurations satisfy
\[
    E_\theta(u^+ \mid h) < E_\theta(u^- \mid h),
\]
where $u^+$ are MILP solutions and $u^-$ are negative samples produced by a sampler.

This induces the model distribution:
\[
    p_\theta(u \mid h)
    = \frac{\exp\!\left( - E_\theta(u \mid h) \right)}
           {Z(h)},
\qquad
Z(h) = \sum_{u} \exp( -E_\theta(u \mid h) ).
\]

\section*{2. Energy Function Architecture}

We construct the input:
\[
    x = \big[\, u \;\|\; h \;\|\; (u \odot h) \,\big],
\]
where:
\begin{itemize}
    \item $u \odot h$ is an elementwise bilinear interaction,
    \item $\|\,$ denotes concatenation.
\end{itemize}

The energy is modeled as a multilayer perceptron:
\[
    E_\theta(u \mid h)
    = f_\theta(x),
\]
with:
\[
    f_\theta(x)
    =
    W_3 \,\phi\!\left(
        W_2 \,\phi\!\left(W_1 x + b_1\right) + b_2
    \right)
    + b_3,
\]
where $\phi$ is GELU or SiLU.

\paragraph{Recommended layer sizes:}
\[
    W_1: (N + 128 + N) \rightarrow 256,\qquad
    W_2: 256 \rightarrow 256,\qquad
    W_3: 256 \rightarrow 64 \rightarrow 1.
\]

\section*{3. Optional Structured Energy Extension}

We may add a quadratic interaction learned from the context:
\[
    E_\theta(u \mid h)
    =
    f_\theta(x)
    +
    u^\top A_\theta(h)\,u,
\]
where $A_\theta(h) \in \mathbb{R}^{N \times N}$ is predicted by a secondary network.

A factorized decomposition is also possible:
\[
    E(u \mid h)
    = \sum_i \phi_i(u_i, h)
    + \sum_{i,j} \psi_{ij}(u_i, u_j, h),
\]
useful to encode ramping, min-up/min-down or storage consistency.

\section*{4. Training Objective: Conditional Contrastive Divergence}

Given positive samples $u^+$ and negative samples $u^-$ obtained from a sampler targeting $p_\theta(u\mid h)$, the objective is:
\[
    \mathcal{L}(\theta)
    =
    \mathbb{E}\!\left[E_\theta(u^+ \mid h)\right]
    -
    \mathbb{E}\!\left[E_\theta(u^- \mid h)\right].
\]

Its gradient is:
\[
    \nabla_\theta \mathcal{L}
    =
    \mathbb{E}\!\left[\nabla_\theta E_\theta(u^+ \mid h)\right]
    -
    \mathbb{E}\!\left[\nabla_\theta E_\theta(u^- \mid h)\right].
\]

Optimization: AdamW, $10^{-4}$ recommended; batch size 32–128.

\section*{5. Gibbs Sampling}

For each coordinate $u_i \in \{0,1\}$:

\[
    p(u_i = 1 \mid u_{-i}, h)
    =
    \frac{
        \exp\!\left( -E_\theta(u_i{=}1, u_{-i} \mid h) \right)
    }{
        \exp\!\left( -E_\theta(u_i{=}1, u_{-i} \mid h) \right)
        +
        \exp\!\left( -E_\theta(u_i{=}0, u_{-i} \mid h) \right)
    }.
\]

Sampling:
\[
    u_i \leftarrow \mathrm{Bernoulli}\!\left( p(u_i=1 \mid u_{-i}, h) \right).
\]

After $K$ iterations, the resulting sample $u^-$ approximates $p_\theta(u \mid h)$.

\section*{6. Discrete SGLD Alternative}

Define continuous dynamics:
\[
    u_{t+1}
    =
    u_t - \eta \nabla_u E_\theta(u_t \mid h)
    + \sqrt{2\eta}\,\xi_t,
\qquad
\xi_t \sim \mathcal{N}(0, I),
\]
then threshold:
\[
    u_i =
    \begin{cases}
        1, & u_i > 0.5,\\
        0, & \text{otherwise}.
    \end{cases}
\]

\section*{7. Integration in the UC/DR/Storage Pipeline}

\begin{enumerate}
    \item MILP oracle provides exact feasible configurations $u^+$.
    \item GNN encoder provides embeddings $h$.
    \item EBM learns $E_\theta(u \mid h)$.
    \item Gibbs/SGLD sampler produces candidate configurations $u^-$.
    \item LP Worker validates feasibility and computes dispatch cost.
\end{enumerate}

\section*{8. Evaluation Metrics}

\begin{itemize}
    \item \textbf{Feasibility Rate}:\; $\Pr[\mathrm{LP}(u^-)~\text{feasible}]$.
    \item \textbf{Energy Gap}:\; $E(u^+) - E(u^-)$.
    \item \textbf{Hamming distance} between $u^+$ and $u^-$.
    \item \textbf{Optimality Gap}: relative cost vs MILP optimum.
    \item \textbf{Mixing Speed}: convergence diagnostics of the sampler.
\end{itemize}

\section*{9. Prototype PyTorch Implementation}

\begin{verbatim}
class EBM(nn.Module):
    def __init__(self, dim_u, dim_h=128):
        super().__init__()
        self.net = nn.Sequential(
            nn.Linear(dim_u + dim_h + dim_u, 256),
            nn.GELU(),
            nn.Linear(256, 256),
            nn.GELU(),
            nn.Linear(256, 64),
            nn.GELU(),
            nn.Linear(64, 1)
        )

    def forward(self, u, h):
        uh = u * h.mean(-1, keepdim=True)
        x = torch.cat([u, h, uh], dim=-1)
        return self.net(x)
\end{verbatim}

\end{document}
